\chapter{Summary}
\label{cha:summary}

\todo{GeoJSON}

\todo{dead-reckoning: what if the user was not on foot?}

\todo{concept of transparency}

\todo{finish} Some new, dedicated technology is needed. All existing smartphones need to be scraped ASAP. As costly, as it might seem, when looking at the available solutions (\cref{sec:existing-uloc}) and overwhelming inconvenience of this method (\cref{sec:user-experience}), this is the only real option. And by ``real'' I mean something that might turn out usable to the ordinary person in not-so-distant future.

It's worth mentioning, that the mediation in this case has little sense, as iBeacons themselves give \emph{exactly} the information we were after: ``which room am I in?'' Answering that question was the main design objective for the iBeacon technology. Thus, instead of having several beacons in one room (317 from the AGH-UST example, \cref{sec:case-agh}), one beacon could be placed in each room under observation. Anyway, this work can be thought of as a research exercise.
