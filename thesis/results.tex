\chapter{Results}
\label{cha:results}

\section{Architecture of the proposed solution}

\section{OpenGIS maps}

\section{Ontology}

\section{Physics-based module for providing alternatives}
\label{sec:physics-module}

\todo{cite Fusion2015.pdf}

\todo[inline]{something about inertial measurements}

\todo{something about beacons} Arguably, this is somewhat of a misuse of the beacons technology. Manufacturers generally speak of only \emph{three} recognizable levels of ``interaction'' with the beacon (based on signal strength): \begin{itemize}
 	\item receiver being in very close proximity to the transmitter,
 	\item receiver being in the same room,
 	\item no interaction at all.
 \end{itemize}
 
The idea for using beacons in the module is to make an assessment about distances to nearby transmitters based on signal strength and law of inverse squares (see \cref{eq:inverse-sq}). Matyasik et al. claim that this works, however, \emph{only under laboratory conditions} \cite{Matyasik:iBeacon}. As iBeacons use extremely low energy to transmit the signal, even a single person walking across the room while taking the measurement---or even someone playing with the smartphone during the measurement---can influence it so negatively, that any information inferred about the signal's distance is almost worthless.

\section{The process of mediation}

\section{User Experience}
\label{sec:user-experience}
