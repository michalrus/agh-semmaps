\chapter{Micro-location}
\label{cha:motivation}

% Motivation

The motivation for providing a solution to micro-location problem is\ldots \todo{motivation}

\section{Overview of existing solutions for micro-location}
\label{sec:existing-uloc}

It is worth noting, that most methods below incorporating electromagnetic signals use the law of inverse square which says that a quantity or---in this case---intensity of some field is inversely proportional to the square of the distance from the source.

\begin{equation}
	\label{eq:inverse-sq}
	I \propto \frac{1}{r^2}
\end{equation}

In this field---pun intended---also called RSSI (or Received Signal Strength Indication) \cite{Gough:RSSI}.

\begin{description}
	\item[Magnetic positioning] is a technique that uses data from smartphone's magnetic sensor, the compass. Most buildings nowadays use reinforced concrete (concrete with \todo{photo}steel bars as a skeleton). These bars affect Earth's magnetic field in a unique way \cite{Haverinen:magnetic}. Crowd sourcing may be used to map a building and, later, these maps may be used to locate a device indoors. This technique gives accuracy of 1--2 meters with confidence level of 90\%. However, it is not really that perfect, as the measurement is heavily impacted by moving metal objects inside the building such as elevators, drawer cabinets etc.
	
	\item[Inertial measurements] is a pedestrian \emph{dead reckoning} technique and as such is a subject to cumulative errors (among others). When used in a smartphone, built-in sensors---accelerometers in all three axes---provide \todo{screenshot}data from which distinct human steps in time and space can be inferred. Later, these steps are overlaid with maps that had to be previously created and user's location is guessed. Even the smallest error made early in this dead reckoning process will manifest itself in every following result, as position $i+1$ is position $i$ with data about $i$-th step added to it. Accelerometers are also pretty noticeable when it comes to battery performance. A lot more energy is being consumed when these sensors are on. Having all these three disadvantages in mind---cumulative errors, battery life and inability to automatically map a previously unknown environment---this is one of the chosen techniques used in physics-based module that provides alternatives for mediation method described in this work (see more about the module in \cref{sec:physics-module}).
	
	\item[Wi-Fi-based positioning] uses information about signal strengths of all \todo{photo}visible Wi-Fi networks at a given moment. From these strengths, using the Received Signal Strength Indication, we can infer distances to all visible access points. Positions of these are known from a previously prepare map. Using these positions and distances, user's location may be easily obtained. Again, it is possible for signal fluctuation to influence the correctness of this method. One notable realization of this method is ``Anyplace'' from researchers at the University of Cyprus \cite{Anyplace}.
	
	\item[iBeacons (via Bluetooth)] are small devices using Bluetooth LE (Low Energy) technology standardized by Apple in 2013. These devices constantly broadcast their unique identifiers to all nearby devices. UUIDs are simply 128-bit numbers usually notated in hexadecimal format, e.g. \texttt{e16d9b56-4d95-48f6-959b-4c8bf701bc6f}. Because of using LE, usually only devices in the same room as the transmitting beacon can receive the signal and by that know what room they're in. Obviously, one has to previously map UUIDs to rooms. Also, not all smartphones currently in use support Low Energy variant of Bluetooth, and thus not all are compatible. This technique is also used as one of the sources in the physics module (\cref{sec:physics-module}).
	
	\item[RFID] is an acronym for radio-frequency identification. This method is similar in its concept to iBeacons. \todo{photo?} Small chips with unique identifiers are placed in the room. RFID reader knows the mapping between chips' IDs and rooms/positions. When an ID is read, its matching room is chosen as the user's current location. One big disqualifying disadvantage is that today mobile phones are not RFID readers.
	
	\item[Grid concepts]
	
	\item[Angle of arrival]
	
	\item[Time of arrival]
	
	\item[Ultrawide band]
	
	\item[Infrared]
	
	\item[Visible light communication]
	
	\item[Ultrasound]
	
	\item[QR]
	
	\item[NFC]
\end{description}

