\chapter{Micro-location}
\label{cha:motivation}

% Motivation

The motivation for providing a solution to micro-location problem is\ldots \todo{motivation}

\section{Overview of existing solutions for micro-location}
\label{sec:existing-uloc}

\begin{description}
	\item[Magnetic positioning] is a technique that uses data from smartphone's magnetic sensor, the compass. Most buildings nowadays use reinforced concrete (concrete with \todo{photo}steel bars as a skeleton). These bars affect Earth's magnetic field in a unique way \cite{Haverinen:magnetic}. Crowd sourcing may be used to map a building and, later, these maps may be used to locate a device indoors. This technique gives accuracy of 1--2 meters with confidence level of 90\%. However, it is not really that perfect, as the measurement is heavily impacted by moving metal objects inside the building such as elevators, drawer cabinets etc.
	
	\item[Inertial measurements] is a pedestrian \emph{dead reckoning} technique and as such is a subject to cumulative errors (among others). When used in a smartphone, built-in sensors---accelerometers in all three axes---provide \todo{screenshot}data from which distinct human steps in time and space can be inferred. Later, these steps are overlaid with maps that had to be previously created and user's location is guessed. Even the smallest error made early in this dead reckoning process will manifest itself in every following result, as position $i+1$ is position $i$ with data about $i$-th step added to it. Accelerometers are also pretty noticeable when it comes to battery performance. A lot more energy is being consumed when these sensors are on. Having all these three disadvantages in mind---cumulative errors, battery life and inability to automatically map a previously unknown environment---this is one of the chosen techniques used in physics-based module that provides alternatives for mediation method described in this work (see more about the module in \cref{sec:physics-module}).
	
	\item[Wi-Fi based positioning]
	
	\item[Bluetooth, iBeacons]
	
	\item[RFID]
	
	\item[Grid concepts]
	
	\item[Angle of arrival]
	
	\item[Time of arrival]
	
	\item[Ultrawide band]
	
	\item[Infrared]
	
	\item[Visible light communication]
	
	\item[Ultrasound]
	
	\item[QR]
	
	\item[NFC]
\end{description}

