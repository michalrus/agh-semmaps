\begin{abstract}
	% What was done?
	This is a paper about a project and proof-of-concept implementation of a mediation system that aids the user in micro-localization of their mobile device by presenting them with certain automatically generated questions.
	
	% Why was it done?
	Most current solutions for micro-localization---at least those applicable to most modern devices currently in use---are faulty in a way that it's extremely difficult to get the exact location correctly, when not measuring in ideal conditions.
	
	% How was it done?
	The questions are generated based on pre-created GML/OpenGIS maps containing semantic information about places/areas that the user can find themselves in. Knowing where the user is most probable to be---this information comes from an external, physics-based micro-localization module---we iteratively generate the best question to distinguish between the available alternatives.
	
	% What was found?
	It was found that this kind of manual labor imposed upon the user significantly improves the quality of micro-localization when compared to the one provided solely by the physical module.
	
	% What is the significance of the findings?
	Hopefully, although not free of hassle for the user---after all, answering these questions is not necessarily a good User Experience---this method can be furthered and deployed in real-world apps, at least until better, more accurate micro-localization methods are implemented in common-use devices.
\end{abstract}
